\documentclass[12pt, a4paper]{article}

% Packages
\usepackage[utf8]{inputenc}
\usepackage{graphicx}
\usepackage{subcaption}
\usepackage{amsmath, amssymb, amsthm}
\usepackage{tikz}
\usepackage{bbm}
\usepackage{hyperref}
\usepackage[authoryear]{natbib}
\usepackage{url}
\usepackage{geometry}
\usepackage{keyval}
\usepackage[french]{babel}
\usepackage[T1]{fontenc}
\usepackage{enumitem}
\usepackage{float}
\usepackage[table]{xcolor}
\usepackage{makecell}
\usepackage[none]{hyphenat}
\usepackage{amsfonts}
\usepackage{algpseudocode}
\usetikzlibrary{positioning}
\usepackage{cancel}
\usepackage{blkarray}
\usepackage{multirow}
\usepackage{fancyhdr}
\usepackage[absolute,overlay]{textpos}

% Page Layout
\geometry{margin=1in}
\pagestyle{fancy}
\fancyhf{}
\fancyhead[R]{\thepage}

\renewcommand*{\sectionmark}[1]{\markright{\thesection.~~#1}}
\renewcommand*{\subsectionmark}[1]{\markright{\thesubsection.~~#1}}
\renewcommand*{\subsubsectionmark}[1]{\markright{\thesubsubsection.~~#1}}
\lhead{\sffamily \rightmark}
\renewcommand{\headrulewidth}{0pt}

% use dots instead of -
\AtBeginDocument{
  \def\labelitemi{$\bullet$}
}
\sloppy

% keys & commands
\input{src/keys.tex}
\input{src/commands.tex}

\newtheorem{theorem}{Theorem}[section]
\newtheorem{lemma}[theorem]{Lemma}
\newtheorem{proposition}[theorem]{Proposition}
\newtheorem{corollary}[theorem]{Corollary}

% Define unnumbered environments
\theoremstyle{definition}
\newtheorem{definition}{Definition}[section]
\newtheorem{example}{Example}[section]
\newtheorem*{remark}{Remark} % No numbering
\setlength{\parindent}{0pt} % remove indent
\numberwithin{figure}{section}
\numberwithin{equation}{section}
\numberwithin{table}{section}

\author{Baye Malick Gning}
\date{}

% TITLE
\title{Gaussian discrimant analysis}


\begin{document}

\maketitle
\vspace{2cm}


\newpage
\tableofcontents
\newpage


% BEGIN
\section{Introduction}
Gaussian discriminant analysis (GDA) is a generative classifier. Compared to Naive Bayes, GDA is restricted to continuous features and assumes features are normally distributed within classes, i.e.,

\begin{equation}
p(X|y=c,\theta)=\mathcal{N}(X|\mu_c, \Sigma_c)
\label{eq:qda-features}
\end{equation}

If $\Sigma_c$ is diagonal, this is equivalent to Naive Bayes. With $k-1+k\left(m+\frac{m(m+1)}{2}\right)$ parameters, GDA has more parameters than Naive Bayes ($k-1+2km$ parameters). Thus, it is more accurate but is the risk of overfitting is higher. In comparison with Naive Bayes, 

\makeimage{
    path=images/qda.png,
    caption=Quadratic discriminant analysis with $k-1+k\left(m+\frac{m(m+1)}{2}\right)$ parameters,
    ref=qda,
    width=0.4
}{H}

As,
% \begin{equation}
% p(y = c | x, \theta) = \frac{p(y = c | \theta) p(x | y = c, \theta)}{\sum\limits_{c} p(y = c | \theta) p(x | y = c, \theta)}
% \end{equation}
\begin{equation}
p(Y = c | X, \theta) = \frac{p(Y = c | \theta) p(X | Y = c, \theta)}{p(X |\theta)}
\label{eq:qda-bayes-formula}
\end{equation}

we can classify a feature vector using the following decision rule:
\begin{equation}
\hat{y}(x) = \arg\max_c \left[ \log p(y = c | \pi) + \log p(x | \theta_c) \right]
\end{equation}

\section{Multivariate normal}
The pdf for an MVN in D dimensions is defined by the following:

\begin{equation}
\mathcal{N}(x | \mu, \Sigma) = \frac{1}{(2\pi)^{D/2} |\Sigma|^{1/2}} \exp \left( -\frac{1}{2} (x - \mu)^T \Sigma^{-1} (x - \mu) \right)
\end{equation}

The expression inside the exponent is the Mahalanobis distance between a data vector $x$ and the mean vector $\mu$. The first two moments are:
\begin{equation}
\mathbbm{E}(X)=\mu \quad \text{and} \quad
\text{cov}(X)=\mathbbm{E}((X-\mu)(X-\mu)^T)
\end{equation}
% We can gain a better understanding of this quantity by performing an eigendecomposition of Σ.

If we have $N$ i.i.d. samples $x_i \sim \mathcal(\mu, \Sigma)$, then the MLE for the parameters is given by:

\begin{equation}
\hat{\mu}_{mle} = \frac{1}{N}\sum_{i=1}^N x_i=\bar{x}
\end{equation}

\begin{equation}
\hat{\Sigma}_{mle} = \frac{1}{N}\sum_{i=1}^N (x_i-x)(x_i-x)^T = \frac{1}{N}\sum_{i=1}^N x_i x_i^T - \bar{x}\bar{x}^T
\end{equation}

\section{MLE for discriminant analysis}
The log-likelihood function is as follows:  

\begin{align}
\log p(\mathcal{D} | \theta)
&= \left[\sum_{i}^{N}\sum_{c=1}^{C} \mathbbm{1}(y_i = c) \log \pi_c\right] + \sum_{c=1}^{C}\left[\sum_{i:y_i = c} \log \mathcal{N}(X|\mu_c, \Sigma_c)\right]
\end{align}

We see that this factorizes into a term for $\pi$, and $C$ terms for each $\mu_c$ and $\Sigma_c$. Hence we can estimate these parameters separately. For the class prior, we have: $\hat{\pi}_c = \frac{N_c}{N}$ as with Naive Bayes with $N_c = \sum_{i} \mathbbm{1}(y_i = c)$ is the number of examples in class $c$. For the class-conditional densities, we just partition the data based on its class label, and compute the MLE for each Gaussian:

\begin{equation}
\hat{\mu}_c = \frac{1}{N_c} \sum_{i: y_i = c} x_i = \frac{1}{N_c} \sum_{i=1}^N \mathbbm{1}(y_i = c) x_i
\end{equation}

\begin{equation}
\hat{\Sigma}_c = \frac{1}{N_c} \sum_{i: y_i = c} (x_i - \hat{\mu}_c)(x_i - \hat{\mu}_c)^T = \frac{1}{N_c} \sum_{i=1}^N \mathbbm{1}(y_i = c) (x_i - \hat{\mu}_c)(x_i - \hat{\mu}_c)^T
\end{equation}

\section{Quadratic discriminant analysis (QDA)}
From equation \ref{eq:qda-bayes-formula}, we derive:
\begin{equation}
p(Y = c | X, \theta) \propto \pi_c \frac{1}{|\Sigma_c|^{1/2}} \exp \left[ -\frac{1}{2} (x - \mu_c)^T \Sigma_c^{-1} (x - \mu_c) \right]
\label{eq:qda-bayes-formula}
\end{equation}
This formula requires $\Sigma_c$ to be well conditioned.

This leads to class boundaries between classes $c_1$ and $c_2$ being quadratic surfaces:
\begin{equation}
\log(\pi{_{c_1}} |\Sigma{_{c_1}}|^{1/2}) - \frac{1}{2} (x - \mu{_{c_1}})^T \Sigma{_{c_1}}^{-1} (x - \mu{_{c_1}})=
\log(\pi{_{c_2}} |\Sigma{_{c_2}}|^{1/2}) - \frac{1}{2} (x - \mu{_{c_2}})^T \Sigma{_{c_2}}^{-1} (x - \mu{_{c_2}})
\label{eq:qda-class-boundaries}
\end{equation}

The result is known as \textbf{quadratic discrimant analysis (QDA)}


\section{Linear discriminant analysis (LDA)}
We now consider a special case in which the covariance matrices are \textbf{tied} or \textbf{shared} across classes (\textbf{homoscedasticity}), $\Sigma_c=\Sigma$. In this case, we can simplify equation \ref{eq:qda-features} as follows:
\begin{equation}
p(X|y=c,\theta)=\mathcal{N}(X|\mu_c, \Sigma)
\end{equation}


LDA has $k-1+km+1\frac{m(m+1)}{2}$ parameters, is less accurate than QDA but has lower risk of overfitting. The MLE for $\Sigma$ is:
\begin{equation}
\hat{\Sigma} = \frac{1}{N} \sum_{c=1}^C \sum_{i:y_i=c} (x_i - \hat{\mu}_c)(x_i - \hat{\mu}_c)^T
\end{equation}


\section{Mahanalobis distance and transformation: another way to interpret LDA}
We performe an eigendecomposition of $\Sigma$. We write  $\Sigma = U \Lambda U^T$ where $U$ is an orthonormal matrix of eigenvectors satisfying $U^T U = I$, and $\Lambda$ is a diagonal matrix of eigenvalues $\text{Diag}(\sigma_i^2)$. Let $\phi$ be the \textbf{Mahalanobis transformation} (rotation $U^T$ + scaling on new axes):  
\begin{equation}
\phi: x \mapsto z = \Sigma^{-\frac{1}{2}} x = \text{Diag}(\sigma_i^{-1}) U^T x
\label{ex:qda-mahanalobis-func}
\end{equation}

The Mahanalobis distance between $x_1$ and $x_2$ can be written in two forms:
\begin{align}
d_M(x_1,x_2)
&=\sqrt{(x_1-x_2)^T \Sigma^{-1} (x_1-x_2)}\\
&=\sqrt{(\phi(x_1)-\phi(x_2))^T (\phi(x_1)-\phi(x_2))}
\label{eq:qda-mahanalobis-distance}
\end{align}

Then $\phi$ decorrelates samples:  
\begin{equation}
\mathbf{x} \sim \mathcal{N}(\mathbf{0}, \Sigma) \quad \Rightarrow \quad \phi(x) \sim \mathcal{N}(0, I)
\end{equation}
% TODO: make program for figure in slide 9
LDA chooses the closest $\mu_c$ in the Mahanalobis space as:

\begin{equation}
p(Y=c|X) \propto \pi_c |\Sigma^{-1}| exp \left[-\frac{1}{2}d_M(X,\mu_c)\right]
\end{equation}


We show below that LDA class boundaries are hyperplanes. Indeed, using $z$, Mahanalobis transformation and inspiring from equation \ref{eq:qda-class-boundaries}, we have:

\begin{align}
&z \in \text{boundary between class 1 and 2}\\
&\Leftrightarrow \frac{1}{2} \|z - z_1\|^2 - \log \pi_1 = \frac{1}{2} \|z - z_2\|^2 - \log \pi_2\\
&\Leftrightarrow \left(z - \frac{z_1 + z_2}{2}\right) (z_2 - z_1) = \log \frac{\pi_1}{\pi_2}
\end{align}

% TODO: code and figures for slide 11, 12, 13


\section{LDA as a dimensionality reduction method}
TODO: slide 10

% TODO: add figure from slides
\section{Strategies for preventing overfitting}
The speed and simplicity of the MLE method is one of its greatest appeals. However, the MLE can badly overfit in high dimensions. In particular, the MLE for a full covariance matrix is singular if $N_c < m$. And even when $N_c > m$, the MLE can be ill-conditioned, meaning it is close to singular. There are several possible solutions to this problem:
\begin{itemize}
\item Use a diagonal covariance matrix for each class, which assumes the features are conditionally independent; this is equivalent to using a naive Bayes classifier
\item Use a full covariance matrix, but force it to be the same for all classes, $\Sigma_c=\Sigma$. This is an example of parameter tying or parameter sharing, and is equivalent to LDA
\item Use a diagonal covariance matrix and forced it to be shared. This is called diagonal covariance LDA
\item Use a full covariance matrix, but impose a prior and then integrate it out. If we use a conjugate prior, this can be done in closed form; this is analogous to the Bayesian Naive Bayes
\item Fit a full or diagonal covariance matrix by MAP estimation.
\item Project the data into a low dimensional subspace and fit the Gaussians there.
\end{itemize}

\section{Regularized LDA}
TODO

\section{Diagonal LDA}
Covariance matrices are tied, so $\Sigma_c=\Sigma$ as in LDA, and then to use a diagonal covariance matrix for each class. The hypothesis can be view as homoscedasticity (LDA) and conditional independence (Naive Bayes). Diagonal LDA has $k-1+km+m$ parameters. Thus, we have even less parameters than Naive Bayes. Diagonal LDA is very simplistic but very robust.

% TODO: code and figure from slide 15 and 16

\section{Shrinkage and high dimensional statistics}
TODO

% TODO: figure of last slide 20



%%%%%%%%%% Appendix
% \newpage
% \section{Appendix}
Here we add some theorems (with demonstrations or not) that may interest the reader.

% TODO: Add ref ML a probabilistic perspective page 119-120 and 
\begin{theorem}[Bayes rule for linear Gaussian systems]
\label{thm:bayes-rule-lin-gaussian-sys}

Given a linear Gaussian system:
\begin{equation}
\boxed{
\begin{aligned}
p(x) &= \mathcal{N} (x | \mu_x , \Sigma_x) \\
p(y | x) &= \mathcal{N} (y | Ax + b, \Sigma_y)
\end{aligned}
}
\end{equation}
the posterior $p(x|y)$ is given by the following:

\begin{equation}
p(x | y) = \mathcal{N} \left( x | \mu_{x | y}, \Sigma_{x | y} \right)
\end{equation}
where:

\begin{equation}
\Sigma_{x | y}^{-1} = \Sigma_x^{-1} + A^T \Sigma_y^{-1} A
\end{equation}

\begin{equation}
\mu_{x | y} = \Sigma_{x | y} \left[ A^T \Sigma_y^{-1} (y - b) + \Sigma_x^{-1} \mu_x \right]
\end{equation}

In addition, the normalization constant $p(y)$ is given by

\begin{equation}
p(y) = \mathcal{N} \left( y | A \mu_x + b, \Sigma_y + A \Sigma_x A^T \right)
\end{equation}
\end{theorem}

\begin{theorem}[Marginals and conditionals of an MVN]
\label{thm:marginals-and-conditionals-mvn}
Suppose \( x = (x_1, x_2) \) is jointly Gaussian with parameters
\begin{equation}
\mu = \begin{pmatrix}
\mu_1 \\
\mu_2
\end{pmatrix}, \quad
\Sigma = \begin{pmatrix}
\Sigma_{11} & \Sigma_{12} \\
\Sigma_{21} & \Sigma_{22}
\end{pmatrix}, \quad
\Lambda = \Sigma^{-1} = \begin{pmatrix}
\Lambda_{11} & \Lambda_{12} \\
\Lambda_{21} & \Lambda_{22}
\end{pmatrix}.
\end{equation}

Then the marginals are given by:
\begin{equation}
p(x_1) = \mathcal{N}(x_1 | \mu_1, \Sigma_{11}), \quad p(x_2) = \mathcal{N}(x_2 | \mu_2, \Sigma_{22}).
\end{equation}

The posterior conditional is given by:
\begin{equation}
p(x_1 | x_2) = \mathcal{N}(x_1 | \mu_{1|2}, \Sigma_{1|2}),
\end{equation}

where

\begin{align}
\mu_{1|2}
&= \mu_1 + \Sigma_{12} \Sigma_{22}^{-1} (x_2 - \mu_2)\\
&= \mu_1 - \Lambda_{11}^{-1} \Lambda_{12} (x_2 - \mu_2)\\
&=\Sigma_{1|2} \left( \Lambda_{11} \mu_1 - \Lambda_{12} (x_2 - \mu_2) \right)
\end{align}

\begin{equation}
\Sigma_{1|2} = \Sigma_{11} - \Sigma_{12} \Sigma_{22}^{-1} \Sigma_{21} = \Lambda_{11}^{-1}.
\end{equation}
\end{theorem}


% TODO: add properties of Gaussian distributions


\newpage
\section{References}
\begin{itemize}
    \item Murphy, K. P. (2012). \textit{Machine Learning: a Probabilistic Perspective}. MIT Press. \url{https://probml.github.io/pml-book/book0.html}
    \item Murphy, K. P. (2023). \textit{Probabilistic Machine Learning: Advanced Topics}. MIT Press. \url{http://probml.github.io/book2}
    \item Statistical Models - SDI Metz
\end{itemize}

\end{document}
